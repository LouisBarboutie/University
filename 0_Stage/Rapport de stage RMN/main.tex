\documentclass{article}
\usepackage[margin=2.5cm]{geometry}
\usepackage[utf8]{inputenc}
\usepackage{hyperref}
\usepackage{amsmath}

\title{Rapport de stage RMN}
\author{Louis-Hendrik Barboutie}
\date{August 2022}


\setlength\parindent{0pt}

\begin{document}

\maketitle

\section{Tagebuch}

\begin{table}[!ht]
    \centering
    \begin{tabular}{|c|l|} \hline
        Tag & Content \\ \hline
         29/08 & lab tour, security tour, starting to study about NMR \\
         30/08 & NEO, further study, looking into rotation frame of reference \\
         31/08 & further study on pulses, free induction decay, fourier transform, test, préparation de l'échantillon d'oxyde de graphène, explication fonctionnement sonde, manipulation du logiciel d'acquisition
    \end{tabular}
    \caption{Caption}
    \label{tab:my_label}
\end{table}

\section{Report}

\subsection{Mercredi 31/08}

La preparation de l'echantillon d'oxide de graphene (OG) se fait dans le spinner de 1.3 mm. Le spinner est assemblé et placé dans un cylindre pour faciliter la manipulation puis l'OG est introduit a l'aide d'une tige pour avoir une bonne quantité de de OG dans l'échantillon. Le rotor est ensuite enfonce comme un capuchon.

L'échantillon est introduit dans le spectomètre via l'ouverture prévue à cet effet. Il faut encore réaliser le "tune et match" de la bobine RF pour la régler à la fréquence de Larmor et d'optimiser son absorbance.

Jésus m'a aussi expliquer quelques détails de la mesure, entre autres le programme contrôlant la séquence d'impulsions, pour provoquer des flips P90 ou P180 afin de réaliser des séquences de spin-echo. Ensuite ont été programmées des sessions successives de mesure pour une durée de 2 jours.

Puisque la sonde est un assemblage de composante électronique (circuit RLC), elle a des caractéristiques propres, influencées par les conditions extérieures, ie. la temperature ou l'échantillon placé dans la bobine. Le signal qu'on cherche à vollecter est très faible, donc il faut ajuster laa frequence de resonnance et l'absorption en fonction des circonstances (cela se fait avec l'analyseur de circuit), notamment dans les cas d'analyses quantitatives.

\subsection{Jeudi 01/09}

Echo de Hahn: Addition du signal pour maximiser rapport signal/bruit. P90 suivi de P180, refocus

Calibration avec le plomb 1ere fois à stras btw: on descend la température à 253K et on calibre l'impulsion: on cherche la temps d'impulsion pour un flip de 90°. ensuite on peut faire des repetitions d'echos de Hahn. Pour le moment 128 répet et on ne tourne pas

en rotation maintenant on fait 2000Hz, et on prend 32 scans, puis on passe à 3000, 4000, 5000 etc. 
Il y a des bandes de rotation car one ne tourne pas assez vite. Un seul pique ne bouge pas et c'est le pic pour l'anisotropie. Il bouge un peu sous l'effet de la température, qui doit croître de façon parabolique. Les bandes de rotations subissent un déplacement cependant. On cherche à déterminer le déplacement chimique pour chaque bande, en prenant la différence de ppm. Ensuite on peut calculer le changement de température avec la formule du paper.

Les scans successifs sont prgrammés: on définit les vitesses de rotations, nombres de séquences d'écho de hahn, ainsi que le temps accordé à l'équilibration thermique, car il y a un gradient l'intérieur de l'échantillon. Pour 64 séquences d'écho on a environ 20min, et donc pour 16 vitesses différentes on met 40min de temps d'équilibration, pour 1h par vitesse, totalant 16h de scans (automatisés).

\subsection{Vendredi 02/09}

Avec la fin des scans programés la veille, il faut maintenant exploiter les données. On mesure le deplacement chimique de la demi-hauteur du pic de droite pour chaque vitesse. Avec ces valeurs on fit une parabole, qui nous donne la relation ppm(rotation) on peut ensuite convertir en température avec la formule du paper, qu'il suffit d'invertir. On fait le calcul d'erreur pour la meilleure précision.

L'aprem on programme les sequences pour tout le weekend. 3 doubles series a des temps d'equilibration differents. Total de 72h de scans. La première série est comme pour le test à 40 min d'équilibration, puis on balaye les vitesses de rotation. La deuxième est à 1h d'équilibration puis balayage et la troisième est à 1h20 d'équilibration puis balayage. Le tout se fait à 255K, stabilisé par le PID.

\subsection{Lundi 05/09}
Exploitation des données des scans.
Group Meeting in the afternoon, Jesus presents Carmen, project on brain tumor study via nmr

\subsection{Mardi 06/09}
Exploitation des données des scans à 273K avec 20 minutes de relaxation. Problème: pas en concordance avec le paper. Pour une difference de temperature de 18K on calcule une difference de 21K.
Beaucoup d'erreur pcq la calibration était fausse.

\subsection{Mercredi 07/09}

Marine montre le remplissage des plus petits spinners. Ensuite Gas flow 3000 l/h pour descendre à 250K et refaire la manip. Exploitation des données as usual. Debut remplissage azote, marine montre remplissage spinner tres petits, casse tete erreur sur la temperature.

\subsection{Jeudi 08/09}

Petit tour chez rpe, rmn mais avec electron et pas noyau. Lancement du statique pour temperature variable. Puisque l'erreur est beaucoup trop grande, on va quand même voir les variations. Refill Helium+ Nitrogen.

\subsection{Vendredi 09/09}
Programming

\subsection{Lundi 12/09}
Programming + lancement des spectres statiques. Toujours le souci du spectre pas beau. Je refais un spinner de plomb, thin-wall donc spectre peut-être différent.

\subsection{Mardi 13/09}
Fin de la manip programmée + évaluation. On descend en step de 10 K de 295 K à 255 K, puis on remonte de 260 K à 290 K en step de 10 K. Problème: la pente des deux courbes n'est pas la même! On a un aspect linéaire tout de même.
Test du premier échantillon: Ca remarche!

\subsection{Mercredi 14/09}

Alex Nevzorov Day: Probe repairs and testing inside the 750MHz spectroscope... pretty boring tbh. Loads of problems with tuning and matching the frequency, so had to swap capacitor, coils etc. until it worked. Optimisation des pulses pour trouver la durée des P90 et P270. Aussi on fait de la CP-MAS pour transferer aimantation de $^1$H à $^{15}$N pour plus de signal

\subsection{Jeudi 15/09}
Réparation sonde: on arrive pas à accorder donc il faut changer des trucs: un filament dans une bobine semble la meilleure solution, brancher une 2e en série aide, mais pas trop. Solution finale est de la connecter à la masse, donc il faut dessouder, dévisser puis reconstruire la sonde. C'est un peu du bricolage.

\subsection{Vendredi 16/09}
Changement Bobine de la sonde, pour un champ B$_1$ plus grand. Fabrication maison. Le diamètre est plus petit et il y a plus de révolutions. Petit souci d'acoustic ringing, où le sample bouge un peu du aux faible fréquences et la force électromotrice

\subsection{Mots-clefs}

\begin{itemize}
    \item Aimantation macroscopique $\Vec{M}$
    \item Spin
    \item Moment magnetique
    \item Champ magnetique externe $\Vec{B_0}$
    \item Precession de Larmor $\frac{d \Vec{M}}{dt} = \gamma \Vec{M} \times \Vec{B_0}$
    \item Frequence de Larmor $\omega_0 = \gamma B_0$
    \item Temps de relaxation (en x,y,z)
    \item Referentiel tournant
    \item Frequence de resonnance
    \item Champ magnetique effectif
    \item Transformee de Fourier
    \item Sequence d'impulsions
    \item Free induction delay
    \item Phase
    \item Zero filling
    \item Inversion recovery 
    \item shim coils -> homogeneiser le champ B0
    \item tune + match : ajuster la frequence de resonnance du circuit avec la bobine pour atteindre la frequence de Larmor
    \item Dwell time
\item Analyseur de circuit.
\end{itemize}


\section{Livres}

\begin{enumerate}
    \item \textit{A complete introduction to modern spectroscopy}, Roger S. Macomber
    \item \url{https://www.cis.rit.edu/htbooks/nmr/inside.htm}
    \item \textit{Understanding NMR Spectroscopy}, James Keeler
\end{enumerate}
\section{Papers}
\begin{enumerate}
    \item Temperature Dependence of 207Pb Spectra of Solid Lead Nitrate. A. Bielecki and D. Burum
    \item Measurement of sample temperatures under magic-angle spinning from the chemical shift and spin-lattice relaxation rate of 79Br in KBR powder. K. Thurber and R. Tycko.
    \item A thermometer of nonspinning solid-state NMR Spectroscopy. P. Beckmann and C. Dybowski
\end{enumerate}

\end{document}
