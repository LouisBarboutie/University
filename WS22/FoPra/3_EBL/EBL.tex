\documentclass[12pt]{article} 
\usepackage[utf8]{inputenc}
\usepackage[margin=2.5cm]{geometry} % decrease the margin size

\usepackage{hyperref} % to include weblinks

\usepackage{graphicx} % handle images and figures
\usepackage{placeins} % to control placement of figures
\usepackage{subcaption}

\usepackage{amsmath} % general package for writing math
\usepackage{amsfonts} % math fonts for obscure symbols

\usepackage[version=4]{mhchem} % chemistry notation

\usepackage[nottoc]{tocbibind}
\usepackage[backend=biber,dateabbrev=false]{biblatex} % package for handling references
\addbibresource{references.bib}
\usepackage{csquotes}

\usepackage{fancyhdr}

\setlength{\parindent}{0pt} % remove indentation at the beginning of pargraphs

\begin{document}

\pagestyle{fancy}
\setlength{\headheight}{15pt}
\fancyhead[L]{Electron Beam Lithography}
\fancyhead[R]{Louis-Hendrik Barboutie, Rajon Bhuyan}

\begin{titlepage}
    \begin{center}
        \vspace*{1cm}
        \Huge
        \textbf{Electron Beam Lithography}
        
        \vspace{0.5cm}
        \LARGE
        Physikalisches Praktikum für Fortgeschrittene I
        
        \vspace{1.5cm}
        \textbf{Louis-Hendrik Barboutie and Rajon Bhuyan} \newline
        \textbf{7016306 \& 7029677}
        
        \vspace{0.5cm}
        \Large 
        Supervisor: Sukhvinder Singh
        
        \vfill

        \includegraphics[width=0.4\textwidth]{logo_uni.png}
        
        \Large
        12$^{\underline{\text{th}}}$ January 2023
    \end{center}
\end{titlepage}

\tableofcontents
\newpage

\section{Answers to questions}

\textbf{Q1. Explain the basic structure and function of an electron microscope! What limits the
resolution of the electron microscope?}

A: An electron microscope is a microscope that uses a beam of accelerated electrons as a source
of illumination. When the electron beam interacts with the specimen, it loses energy by a
variety of mechanisms. The lost energy is converted into alternative forms such as heat,
emission of low energy secondary electrons and high-energy backscattered electrons, light
emission (cathodoluminescence) or X-ray emissions, all of which provide signals carrying
information about the properties of the specimen surface, such as its topography and
composition.
Resolution of the electron microscope can be limited by the objective lens system in electron
microscopes (the aperture cannot be made very small that the current is undetectable), stability
of electron beam and magnetic field, and quality of vacuum in the tube. In addition to that
diffraction of e- beam and astigmatism also limits the resolution.

\textbf{Q2. How are X-rays generated? In which wavelength range are X-rays?}

A: X-Rays are generated when core electrons of an atom are removed through collision. X- rays
can also be produced through Synchrotron radiation, where charged particles are subject to an
acceleration perpendicular to their velocity.
X-rays have a wavelength ranging from 10 picometers to 10 nanometers.

\textbf{Q3. Which different operating modes of an electron microscope work? Explain how they
work and explain the differences?}

A: Electron microscopes work in the following modes:
Transmission electron microscope (TEM)
A high energy beam of electrons is shone through a very thin sample, and the interactions
between the electrons and the atoms can be used to observe features of the sample. The beam
strikes the sample and parts of it are transmitted depending upon the thickness and electron
transparency of the specimen. This transmitted beam is then focused into an image on
fluorescent screen.
Scanning electron microscope (SEM)
Instead of travelling through the sample, the e- beam strikes the surface of the sample which is
coated with a thin layer of Au or Pd. The electrons strike bound electrons of the coated layer
which produces backscattered electrons and secondary electrons. When the secondary
electrons reach and enter the detector, they strike a scintillator. This emits flashes of light is
then used to create an image. The raised surfaces appear brighter on the screen while depressed
surfaces appears darker.
Electron beam lithography

\textbf{Q4. Explain the advantages and disadvantages of the different types of lithography. What is
the difference between raster and vector beam steering? What is meant by meander or
line mode?}

In raster beam steering, the beam is scanned over the sample line by line. Wherever the sample
needs not be exposed, the beam is blanked using a slit diaphragm.
In vector beam steering, the beam is scanned over the sample where exposure is required and
blanking is kept minimal by finding the least travel distance through a software.

Meander mode (MM mode): The electron beam meanders along each line to be exposed in
alternating writing direction.
Line mode (LM mode). The electron beam travels along only one direction. The beam is
blanked while travelling back to the starting line.

\textbf{Q5. What is meant by a positive or negative resist? What type of PMMA is used here?}

A: When positive photoresist is exposed to radiation, the chemical structure changes and becomes
more soluble in the photoresist developer. These exposed areas are then washed away with the
photoresist developer solvent.
With negative photoresist, instead of becoming more soluble, negative photoresists become
extremely difficult to dissolve. As a result, negative resist remains on the surface while the
photoresist developer solution works to remove the areas that are unexposed.
PMMA used here is a positive resist consisting of long-chain monomers whose bonds are
broken as a result of electron beam exposure.

\textbf{Q6. Explain the main reaction when the PMMA layers are exposed to light?}

A: In positive resists, PMMA polymer chains are broken into subsequent monomers.
In negative resists, PMMA polymer chains are broken into monomers and then repolymerises
through cross-linking.

\textbf{Q7. Which secondary processes are triggered by the impact of the electron beam on the
sample surface?}

A: When electrons strike the sample surface, electrons are ejected from atoms which are called
Auger electrons. Some electrons scatter back along the beam path and interactions with the
substrate generate X-rays.
Electron beam lithography

\textbf{Q8. What is the Proximity Effect? Explain the effects during exposure of the sample.}

A: The proximity effect in electron beam lithography (EBL) is the phenomenon that the exposure
dose distribution, and hence the developed pattern, is wider than the scanned pattern due to the
interactions of the primary beam electrons with the resist and substrate. These cause the resist
outside the scanned pattern to receive a non-zero dose.
During exposure forward scattering and backscattering takes
place. The forward scattering process is due to electron-electron
interactions which deflect the primary electrons, broadening the
beam in the resist. The electrons do not stop in the resist but
penetrate the substrate. And contribute to resist exposure by
scattering back into the resist. This backscattering process
originates from a collision with substrate nucleus and leads to
wide-angle scattering of the light electron from a range of depths
(micrometres) in the substrate.

\textbf{Q9. Explain the difference between intra- and inter-proximity effects?}

A: During exposure of a substrate to an electron beam, due to different lateral dimensions getting
exposed to same dose of radiation, underexposure can take place. This is intra-proximity effect.
Inter-proximity effect takes place when two adjacent structures are exposed and there is
overexposure. This inter-proximity effect results in bulging of the respective edge areas
towards the neighbouring structure.

\textbf{Q10. Name different methods for determining the proximity parameters and explain them.}

A: Underexposure (intra-proximity) can be determined with the smaller structures, where the
scatter components are significantly lower.
Inter-proximity can be determined be exposing two adjusting structures with decreasing
spacing between them. The overexposure effect that sets in depending on the dose and the
minimum distance is clearly visible.
The structures are then analysed using light microscopy or SEM

\section{Results}

\section{Discussion}

\printbibliography

\end{document}