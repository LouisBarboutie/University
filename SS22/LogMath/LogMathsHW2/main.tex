\documentclass{article}
\usepackage[utf8]{inputenc}
\usepackage[margin=2.5cm]{geometry}
\usepackage{amsmath}
\usepackage{amssymb}
\usepackage{amsthm}

\setlength\parindent{0pt}

\newtheorem{theorem}{Theorem}

\newtheorem{proposition}[theorem]{Proposition}

\theoremstyle{definition}
\newtheorem*{remark}{Remark}

\title{LogMathsHW2}
\author{Louis-Hendrik Barboutie}
\date{April 2022}

\begin{document}

\maketitle

\begin{proposition}[Fundamental Theorem of Algebra]
    Let p(x) be a non-constant polynomial with coefficients in $\mathbb{C}$. Then there is $z \in \mathbb{C}$ such that $p(z) = 0$.
\end{proposition}

\begin{remark}
    Proposition 1 is not true for polynomials with coefficients in $\mathbb{R}$. For example 
    \begin{equation}
        p(x) = x^2 + 1
    \end{equation}
    does not have real roots.
\end{remark}

\begin{theorem}
    If X and Y are $\sigma$-finite measure spaces and $f: X \times Y \rightarrow \mathbb{R}$ is measurable and such that 
    \begin{equation*}
        \int_{X  \times Y}|f(x,y)|d(x,y) < \infty
    \end{equation*}
    then
    \begin{equation}
        \int_X \left( \int_Y f(x,y)dy \right)dx = \int_Y \left( \int_X f(x,y)dx \right)dy = \int_{X  \times Y}f(x,y)d(x,y).
        \label{eq:someEq}
    \end{equation}
\end{theorem}

\begin{remark}
    In practice, equation~(\ref{eq:someEq}) means that we can switch the order of integration in a double integral.
\end{remark}


\end{document}
