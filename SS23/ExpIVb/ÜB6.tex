\documentclass[12pt]{article}
\usepackage[margin=2.5cm]{geometry}
\usepackage{graphicx}
\usepackage{amsmath,amssymb,amsfonts}
\usepackage[version=4]{mhchem}
\usepackage{siunitx}
\usepackage{fancyhdr}

\pagestyle{fancy}
\setlength{\headheight}{16pt}
\fancyhead[L]{Übungsblatt 6}
\fancyhead[R]{Louis-Hendrik Barboutie}

\setlength{\parindent}{0pt}

\begin{document}

\section{Aufgabe 15}

\subsection{a}

Besteht 1 \si{T} Holzkohle zu 90 \% aus Carbon, entspricht dies einer Masse $M_C = 900 \ \si{kg}$ Carbon.

Die Gesamtmasse Carbon setzt such außerdem zusammen aus 
\begin{equation*}
    M_C = M_{C-12} + M_{C-14}
\end{equation*}

\end{document}